\documentclass[]{article}

\usepackage{graphicx}
\usepackage{xcolor}
\usepackage{pgfplots}
\usepackage{amsmath}
\pgfplotsset{compat=1.18}

%opening
\title{\textbf{Traductor de C a ensamblador \\ - \\ Procesadores de Lenguajes}}
\author{Rafael Caro de los Reyes \\ Alejandro García Ramos \\ Adrián Muñoz López \\\\ - Estudiantes de la Universidad de Cádiz - \\\\ rafa.carodelosreyes@alum.uca.es \\ ale.garciara@alum.uca.es \\ adrian.munozlopez@alum.uca.es}

\begin{document}
	
	\maketitle
	
	\begin{abstract}
	En este proyecto se lleva a cabo el desarrollo de una aplicación que permite traducir un código fuente en el lenguaje de programación C a ensamblador. A lo largo de este documento se especificaran las capacidades del traductor, detallando desde cómo debe de ser el código fuente que es capaz de traducir a los problemas que tenga el traductor si los tuviera.\\ \\
	\end{abstract}
	
	\section{Introducción}
        Antes de comenzar, es importante destacar que el traductor está desarrollado utilizando una librería de python llamada \texttt{\textbf{Sly}}. En nuestro proyecto hemos modularizado el traductor en 5 archivos principales: \\ \begin{itemize}
        \item \textbf{main.py:} Es el archivo principal que se usa para lanzar el programa.
        \item \textbf{CLexer.py:} Como el nombre indica, contiene todo el código relacionado con el Lexer.
        \item \textbf{CParser.py:} Como el nombre indica, contiene todo el código relacionado con el Parser.
        \item \textbf{clasesnodos.py:} Contiene el código relacionado con la semántica del traductor.
        \item \textbf{asm\_{translator.py:}} Contiene todo el código relacionado con la traducción a ensamblador. \\ \\
        \end{itemize}

        \section{Capacidades}
        En esta sección hablaremos del código fuente en C con el que es capaz de trabajar el traductor divido por las secciones en base a las diferentes prácticas que se han trabajado en el curso de \texttt{Procesadores de Lenguajes}. \\\\ Como referencia, se puede acceder en el código fuente adjunto junto a esta documentación a una carpeta llamada \texttt{\textbf{tests}} en la que se encuentra un código en C de ejemplo que demuestra el funcionamiento de cada una de los subapartados. \\\\ Las dos principales restricciones de nuestro traductor son: \\ \begin{itemize}
        \item Para poder realizar cualquier traducción debe de existir en el código fuente al menos una función.
        \item No puede haber ningún comentario al inicio del código fuente. \\
        \end{itemize}

        \subsection{Variables globales}
        Hemos determinado que, para simplificar la traducción, las variables son inicalizadas siempre al inicio de la función, esto no genera ningún problema debido a que siempre debe de existir al menos una función. \\\\ \textbf{Nota:} Como referencia traducir el archivo \texttt{\textbf{tests\textbackslash variables\_{globales.c}}}.

        \subsection{Funciones}
        
        
\end{document}
